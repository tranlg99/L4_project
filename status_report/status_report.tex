    
\documentclass[11pt]{article}
\usepackage{times}
    \usepackage{fullpage}
    
    \title{3D Particle-Based Video Prediction}
    \author{Linda Tran - 24721235T }

    \begin{document}
    \maketitle
 

\section{Status report}

\subsection{Proposal}\label{proposal}

\subsubsection{Motivation}\label{motivation}
The task of video frame prediction is of high relevance to many computer vision applications, particularly to robotics or self-driving cars. However, current deep-learning methods rely on high volume of labeled data which is hardly at hand. This project aims to elevate this problem by using a new "particle video" approach for tracking pixels (Persistent Independent Particles - PIPs) to obtain training data, and create a simple prediction model based on the data.

\subsubsection{Aims}\label{aims}
This project will develop a prediction model that takes a single video frame as input, and outputs a point cloud representing the coordinates of tracked pixels in the future frame. Potentially, the model will reconstruct future frames from the predicted point cloud.

\subsection{Progress}\label{progress}
\begin{itemize}
\itemsep0em
    \item Dataset source chosen (YouTube-8M dataset) but the category might change.
    \item Implemented tracking of pixels with PIPs.
    \item Created pipelines for downloading videos.
    and creating frames from YouTube-8M dataset, generating training data using PIPs and storing training data in Google Drive.
    \item Defined dataset class structure to fit PyTorch Dataloader.
    \item Created prediction neural network by modifying FCN Resnet50 semantic segmentation model.
\end{itemize}    

\subsection{Problems and risks}\label{problems-and-risks}
\subsubsection{Problems}\label{problems}
\begin{itemize}
\itemsep0em
    \item Performance of PIPs particle tracking was not as good as expected, especially on videos with homogeneous backgrounds, which has limited the choice of appropriate datasets.
    \item Limited Colab GPU memory and Drive storage space.
    \item Acquiring training data takes a significant amount of time.
\end{itemize}   

\subsubsection{Risks}\label{risks}
\begin{itemize}
\itemsep0em
    \item Uncertain if the chosen dataset is appropriate both for prediction and PIPs - very small motion, too much noise in the data. \textbf{Mitigation:} Good automated pipeline to easily change to a different one.
    \item Generating the whole training dataset will take a substantial amount of time and Colab has limited running time. \textbf{Mitigation:} Use university's stlinux servers with GPUs, generate training data in chunks.
\end{itemize}  

\subsection{Plan}\label{plan}
\begin{itemize}
\itemsep0em
    \item \textbf{December}
            \begin{itemize}
                \itemsep0em
                \item Add visibility of pixels into training data
                \item Get a small training dataset (10 videos)
                \item Fine-tune the prediction model on small dataset
            \end{itemize}
    \item \textbf{January}
            \begin{itemize}
                \itemsep0em
                \item Acquiring the whole training dataset
                \item Fine-tune the prediction model on the whole dataset
                \item Evaluate the model
            \end{itemize}
    \item \textbf{February}
            \begin{itemize}
              \itemsep0em
              \item Reconstruct the future frames
              \item Figure out how to eliminate 'holes' introduced in the future frames
            \end{itemize}
    \item \textbf{March}
            \begin{itemize}
                \itemsep0em
                \item Dissertation write up
            \end{itemize}
\end{itemize}

\subsection{Ethics and data}\label{ethics}
This project does not involve human subjects or data. No approval required.

\end{document}
